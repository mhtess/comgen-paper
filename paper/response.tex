%----------------------------------------------------------------------------------------
%	PACKAGES AND OTHER DOCUMENT CONFIGURATIONS
%----------------------------------------------------------------------------------------
 
\documentclass[11pt,letterpaper]{letter} % Specify the font size (10pt, 11pt and 12pt) and paper size (letterpaper, a4paper, etc)
%\usepackage[urw-garamond]{mathdesign}
\usepackage{graphicx} % Required for including pictures
\usepackage{microtype} % Improves typography
%\usepackage{gfsdidot} % Use the GFS Didot font: http://www.tug.dk/FontCatalogue/gfsdidot/
\usepackage[T1]{fontenc} % Required for accented characters
\usepackage{url}
% Create a new command for the horizontal rule in the document which allows thickness specification
\makeatletter
\def\vhrulefill#1{\leavevmode\leaders\hrule\@height#1\hfill \kern\z@}
\makeatother
 
%----------------------------------------------------------------------------------------
%	DOCUMENT MARGINS
%----------------------------------------------------------------------------------------

\textwidth 7in
\textheight 9.5in
\oddsidemargin -.25in
\evensidemargin -.25in
\topmargin -1in
\longindentation 0.50\textwidth
\parindent 0.4in
 
%----------------------------------------------------------------------------------------
%	SENDER INFORMATION
%----------------------------------------------------------------------------------------

\def\Who{} % Your name
\def\What{} % Your title
\def\Autha{Michael Henry Tessler} % Your name
\def\Authb{Noah D. Goodman} % Your name
\def\Where{Department of Psychology} % Your department/institution
\def\Address{450 Serra Mall, Bldg 420} % Your address
\def\CityZip{Stanford, CA 94305} % Your city, zip code, country, etc
\def\Email{E-mail: mtessler@stanford.edu} % Your email address
\def\TEL{Phone: (757) 561-7971} % Your phone number



%----------------------------------------------------------------------------------------
%	HEADER AND FROM ADDRESS STRUCTURE
%----------------------------------------------------------------------------------------

\address{
\includegraphics[width=0.8in]{logo-simple.pdf} % Include the logo of your institution
\hspace{5.1in} % Position of the institution logo, increase to move left, decrease to move right
\vskip -0.85in~\\ % Position of the text in relation to the institution logo, increase to move down, decrease to move up
\Large\hspace{1.5in}STANFORD UNIVERSITY \hfill %~\\[0.05in] % First line of institution name, adjust hspace if your logo is wide
%\hspace{1.5in}University 
\hfill 
\normalsize % Second line of institution name, adjust hspace if your logo is wide
\makebox[0ex][r]{\bf \Who \What }\hspace{0.08in} % Print your name and title with a little whitespace to the right
~\\[-0.11in] % Reduce the whitespace above the horizontal rule
\hspace{1.5in}\vhrulefill{1pt} \\ % Horizontal rule, adjust hspace if your logo is wide and \vhrulefill for the thickness of the rule
\hspace{\fill}\parbox[t]{2.85in}{ % Create a box for your details underneath the horizontal rule on the right
%\flushright 
\footnotesize % Use a smaller font size for the details
\flushright
%\Who \\  % Your name, all text after this will be italicized
\Autha\\
\Authb\\
\Where\\ % Your department
\Address\\ % Your address
\CityZip\\ % Your city and zip code
%\TEL\\ % Your phone number
%\Email\\ % Your email address
%\URL % Your URL
}
\hspace{0in} % Horizontal position of this block, increase to move left, decrease to move right
\vspace{-1in} % Move the letter content up for a more compact look
}

%----------------------------------------------------------------------------------------
%	TO ADDRESS STRUCTURE
%----------------------------------------------------------------------------------------

\def\opening#1{\thispagestyle{empty}
{\centering\fromaddress \vspace{0.6in} \\ % Print the header and from address here, add whitespace to move date down
\hspace*{\longindentation}\hspace*{\fill}\par} % Print today's date, remove \today to not display it
%{\raggedright \toname \\ \toaddress \par} % Print the to name and address
\vspace{0.4in} % White space after the to address
\noindent #1 % Print the opening line
% Uncomment the 4 lines below to print a footnote with custom text
%\def\thefootnote{}
%\def\footnoterule{\hrule}
%\footnotetext{\hspace*{\fill}{\footnotesize\em Footnote text}}
%\def\thefootnote{\arabic{footnote}}
}

%----------------------------------------------------------------------------------------
%	SIGNATURE STRUCTURE
%----------------------------------------------------------------------------------------

\signature{\Autha \\ \Authb \What} % The signature is a combination of your name and title

\long\def\closing#1{
\vspace{0.1in} % Some whitespace after the letter content and before the signature
\noindent % Stop paragraph indentation
\hspace*{\longindentation} % Move the signature right
\parbox{\indentedwidth}{\raggedright
#1 % Print the signature text
\vskip 0.1in % Whitespace between the signature text and your name
\fromsig}} % Print your name and title

%----------------------------------------------------------------------------------------

\usepackage{color}
\newcommand{\ndg}[1]{{\color{green}{[ndg: #1]}}}
\newcommand{\mht}[1]{{\color{blue}{[mht: #1]}}}


\begin{document}

\begin{letter}

\opening{Dear Professor Saxe,}

Thank you for soliciting peer reviews for our manuscript \emph{The Language of Generalization}. We appreciate the time both you and the reviewers have devoted to commenting and critiquing our manuscript. We are grateful for the opportunity to resubmit a revised version to be considered for publication in \emph{Psychological Review}.
We have made a number of substantive changes in this version of the manuscript that we hope will improve the readability of the paper.

Inspired by Reviewer 1's request to see the presentation of the various puzzles in a more accessible form,
we have merged the General Discussion subsection ``Other philosophical puzzles'' with the worked examples (e.g., \emph{birds lay eggs vs. are female}, \emph{mosquitos carry malaria}) that appeared before the experiments; now, both appear before the experiments.
We describe why each example is interesting, how the model handles each case, and provide a summary in a table (per R1's suggestion).
\textbf{We elaborated the puzzle of ``Elephants live in Africa and Asia'' by adding a figure of the model predictions for this example.
We note in the worked examples when the relevant puzzle will be tested empirically (e.g., we now foreshadow the relevance of \emph{predictive probability} and motivate Expt. 2C using ``Supreme Court Justices have even social security numbers'').}

We have taken Reviewers 1 and 3's suggestion of reconfiguring some of the general discussion material so now a comparison to other theories appears before the experiments (after the model section and worked examples).
We have also tightened this discussion, hopefully addressing R3's uncertainty about the relationship of our theories to extant theories. 
Our theory is, in R3's option-space, either (e) an outright alternative to prior theories, or (d) a more precise specification of (at least some aspects of) prior theories, namely Cohen (1999)'s domain-restriction theory and Leslie (2007)'s cognitively-based context sensitivity.
This equivocation comes because these previous theories admit different interpretations, only some of which are compatible with our model.
This is hopefully now more clearly articulated.

In the hope of addressing Reviewer 3's second comment, we include an additional subsection in the ``alternative theories'' section describing the various alternative quantitative models we will use for model comparison. 
The importance of these quantitative comparisons is slightly orthogonal to the comparisons to Leslie, Cohen, and other linguistic theories: These quantitative comparisons provide assurance that our model is not overparameterized and that the prior in our model is not doing all of the work. 
This also allows us to clarify the sense in which our model is unique (R3's third comment): It is the first model of generics that makes quantitative predictions about human endorsements (and it does so by connecting with a truth-conditional semantic theory in a particular way). 

\textbf{Reviewer 3 also made the more general comment that the presentation of material meanders in the paper. 
We agree and have added a brief ``Intermediate summary'' section between the two halves of the paper (i.e., the model and related theories section vs. the experiments). 
This section summarizes the model, worked examples, and related theories and provides an overview of the experiments and the utility of each.
We hope this aids the readability.}

We tremendously appreciate Reviewer 2's thoughtful exposition on our account and have incorporated a number of their insights and suggestions into the General Discussion.
We have added a paragraph to the introductory remarks of the General Discussion about how accepting a generic sentence adds to the common ground information about cue validity, which suggests other empirical tests of the accounts.
We have noted as R2 does that there have been no empirical tests of the analogy between habituality and genericity, and that our formal machinery shows the way in which the two can be thought of as the same kind of linguistic phenomena.
We have added a subsection to the General Discussion to clarify the kind of vagueness we have in mind for generics, namely what R2 had in mind: vagueness and genericity are analogous linguistic phenomena insofar as they can both be thought of as the result of a listener being uncertainty as to the exact truth conditions. 
We note that, from a modeling perspective, this sort of ``lexical uncertainty`` is kind of \emph{parameter learning} problem (e.g., the listener knows the form of the truth conditions, but not the values of some variables) which can be formally distinguished from other sorts of uncertain truth conditions better modeled as a \emph{structure learning} problem (e.g., lexical ambiguity), \textbf{which might point the way towards a more general information-theoretic typology of underspecification. }
We clarify that the confluence of underlying mechanisms between generic vagueness and adjectival vagueness doesn't imply that generic vagueness has all the features of the adjectival kind (which indeed it probably doesn't -- for instance sorites sequences).
We additionally now discuss one interpretation of the uncertain threshold, drawing a connection explicitly to Williamson and Simons (1992)'s epistemicist view.\footnote{
	R2 actually cited Williamson (1994). In our literature search, however, we could only find a 1998 book chapter published in the Encyclopedia of Cognitive Science, a book published in 2002, both under the title ``Vagueness'', and a paper in 1992 called ``Vagueness and Ignorance'' written with Peter Simons. 
}

In addition to these broad theoretical points, we have tried to clarify a number of smaller points brought up by Reviewer 1.
We have been more explicit in our relation of generics with quantifiers as being from a purely semantic perspective, not taking into account pragmatic interpretations
We have removed our argument from analogy about states of uncertainty being more fundamental than states of certainty.
That comment was meant in a very particular way relating to Leslie (2007; 2008)'s theory and rhetoric; we now express the same idea by noting that our uncertain threshold model can be viewed as a formalization of Leslie's notion of a \emph{default generalization}. 
We have maintained our discussion of the acquisition of the semantics of generics because it has been a topic of interest to developmental psychologists (and central to Leslie's account), but we have cleaned up the prose and made explicit the way in which we believe learning generic language facilitates learning quantified semantics (in short: they share the same logical form, but the semantics of generics are simpler because they lack a fixed threshold).
\textbf{Reviewer 1 also brings to our attention to property-based theories of concepts. We now are explicit---in the section on Intuitive Theories in the General Discussion---that the fact that our semantics is defined in terms of individual properties does not imply that the concepts involved in understanding generics are defined merely by their properties. Rather, complex intuitive theories, which can reflect conceptual role, produce prevalence priors which then yield generic judgments.}

Finally we have gone through the manuscript and tried to make small changes to improve clarity and citations.

%things i'm not sure about:
%\begin{enumerate}
%\item property theories of meaning (R1) \ndg{they are alluding to theories of concepts -- conceptual role, feature-based, etc. i think we can add a few comments to the section on relation to conceptual theories of generics about this. basically we are not committed to any particular theory of concepts, as long as they are able to yield quantitative feature probabilities.} \mht{i didn't feel completely confident writing about theories of concepts and don't have a sense of ``standard`` citations that would convey succinctly our point. can you add this?}
%\end{enumerate}


\closing{Sincerely yours,}

\end{letter}

\end{document}

