%----------------------------------------------------------------------------------------
%	PACKAGES AND OTHER DOCUMENT CONFIGURATIONS
%----------------------------------------------------------------------------------------
 
\documentclass[11pt,letterpaper]{letter} % Specify the font size (10pt, 11pt and 12pt) and paper size (letterpaper, a4paper, etc)
%\usepackage[urw-garamond]{mathdesign}
\usepackage{graphicx} % Required for including pictures
\usepackage{microtype} % Improves typography
%\usepackage{gfsdidot} % Use the GFS Didot font: http://www.tug.dk/FontCatalogue/gfsdidot/
\usepackage[T1]{fontenc} % Required for accented characters
\usepackage{url}
% Create a new command for the horizontal rule in the document which allows thickness specification
\makeatletter
\def\vhrulefill#1{\leavevmode\leaders\hrule\@height#1\hfill \kern\z@}
\makeatother
 
%----------------------------------------------------------------------------------------
%	DOCUMENT MARGINS
%----------------------------------------------------------------------------------------

\textwidth 6.55in
\textheight 9.25in
\oddsidemargin -.25in
\evensidemargin -.25in
\topmargin -1in
\longindentation 0.50\textwidth
\parindent 0.4in
 
%----------------------------------------------------------------------------------------
%	SENDER INFORMATION
%----------------------------------------------------------------------------------------

\def\Who{} % Your name
\def\What{} % Your title
\def\Autha{Michael Henry Tessler} % Your name
\def\Authb{Noah D. Goodman} % Your name
\def\Where{Department of Psychology} % Your department/institution
\def\Address{450 Serra Mall, Bldg 420} % Your address
\def\CityZip{Stanford, CA 94305} % Your city, zip code, country, etc
\def\Email{E-mail: mtessler@stanford.edu} % Your email address
\def\TEL{Phone: (757) 561-7971} % Your phone number



%----------------------------------------------------------------------------------------
%	HEADER AND FROM ADDRESS STRUCTURE
%----------------------------------------------------------------------------------------

\address{
\includegraphics[width=0.8in]{logo-simple.pdf} % Include the logo of your institution
\hspace{5.1in} % Position of the institution logo, increase to move left, decrease to move right
\vskip -0.85in~\\ % Position of the text in relation to the institution logo, increase to move down, decrease to move up
\Large\hspace{1.5in}STANFORD UNIVERSITY \hfill %~\\[0.05in] % First line of institution name, adjust hspace if your logo is wide
%\hspace{1.5in}University 
\hfill 
\normalsize % Second line of institution name, adjust hspace if your logo is wide
\makebox[0ex][r]{\bf \Who \What }\hspace{0.08in} % Print your name and title with a little whitespace to the right
~\\[-0.11in] % Reduce the whitespace above the horizontal rule
\hspace{1.5in}\vhrulefill{1pt} \\ % Horizontal rule, adjust hspace if your logo is wide and \vhrulefill for the thickness of the rule
\hspace{\fill}\parbox[t]{2.85in}{ % Create a box for your details underneath the horizontal rule on the right
%\flushright 
\footnotesize % Use a smaller font size for the details
\flushright
%\Who \\  % Your name, all text after this will be italicized
\Autha\\
\Authb\\
\Where\\ % Your department
\Address\\ % Your address
\CityZip\\ % Your city and zip code
%\TEL\\ % Your phone number
%\Email\\ % Your email address
%\URL % Your URL
}
\hspace{0in} % Horizontal position of this block, increase to move left, decrease to move right
\vspace{-1in} % Move the letter content up for a more compact look
}

%----------------------------------------------------------------------------------------
%	TO ADDRESS STRUCTURE
%----------------------------------------------------------------------------------------

\def\opening#1{\thispagestyle{empty}
{\centering\fromaddress \vspace{0.6in} \\ % Print the header and from address here, add whitespace to move date down
\hspace*{\longindentation}\hspace*{\fill}\par} % Print today's date, remove \today to not display it
%{\raggedright \toname \\ \toaddress \par} % Print the to name and address
\vspace{0.4in} % White space after the to address
\noindent #1 % Print the opening line
% Uncomment the 4 lines below to print a footnote with custom text
%\def\thefootnote{}
%\def\footnoterule{\hrule}
%\footnotetext{\hspace*{\fill}{\footnotesize\em Footnote text}}
%\def\thefootnote{\arabic{footnote}}
}

%----------------------------------------------------------------------------------------
%	SIGNATURE STRUCTURE
%----------------------------------------------------------------------------------------

\signature{\Autha \\ \Authb \What} % The signature is a combination of your name and title

\long\def\closing#1{
\vspace{0.1in} % Some whitespace after the letter content and before the signature
\noindent % Stop paragraph indentation
\hspace*{\longindentation} % Move the signature right
\parbox{\indentedwidth}{\raggedright
#1 % Print the signature text
\vskip 0.1in % Whitespace between the signature text and your name
\fromsig}} % Print your name and title

%----------------------------------------------------------------------------------------

\begin{document}

\begin{letter}

\opening{Dear Editor,}

We enclose the manuscript \emph{The Language of Generalization} to be considered for publication in \emph{Psychological Review}. 
This paper explores the linguistic means by which generalizations are communicated.  
We present a Bayesian model of understanding generalizations and show that it predicts context-sensitive human endorsement judgments with high quantitative accuracy across diverse experimental contexts.
The theory suggests that the language of generalization can be understood in terms of a simple semantic theory that operates over beliefs, represented as probability.

Understanding how the world works requires learning generalizations. 
The kinds of observations that would facilitate such learning, however, may be rare (e.g., figuring out that lightning strikes tall objects) or costly (e.g., learning that certain plants are poisonous).
We do not need to need taste every plant to figure out which ones are poisonous, however.
Language lets us learn complex and abstract generalizations from each other.

The language of generalization is most commonly studied in predications of categories (e.g.,~``Birds fly.''), otherwise known as \emph{generic language}.
But generalization exists beyond classically-defined natural and artifact kinds: Events are described in generalization (e.g.,~``Mary swims after work.''; \emph{habitual language}) as well as causal relationships (e.g.,~``Fire causes smoke''; \emph{causal language}).
Generalizations are ubiquitous in everyday conversation (``Movie theaters are cold.''), the speech of young children (``Slides aren't for grown ups!''), child-directed speech (``Dogs are friendly.''), political and scientific discourse (``Politicians make promises.'', ``Psychology experiments don't replicate.''), motivation (``Big girls eat broccoli!''), stereotypes (``Boys are good at math.''), and many other facets of experience.


%Given that inductive generalization can be described by probability (e.g., the probability that the next instance of the category will have a feature), it seems natural that the core meaning generalizations in language would relate to this predictive probability.
Despite their ubiquity and morphosyntactic simplicity, the language of generalization has a philosophically puzzling meaning, which has resisted precise formalization. 
%However, the language of generalizations displays extreme sensitivities to context, which has heretofore crippled previous attempts to formalize this major function of language. 
%Though this type of language is found everywhere, the meaning of generalization is philosophically puzzling and has resisted precise formalization. 
%The major issue in formalizing the language of generalizations is determining what makes such statements true or false.
For example, true generic statements can describe properties contained by a majority of the category (e.g., ``Dogs bark.''), roughly half of the category (e.g., ``Robins lay eggs.'' is only true of adult, female, fertile robins), and even an outstanding minority (e.g., ``Mosquitos carry malaria.'' is true of less than 1\% of mosquitoes). 
Habitual and causal language display similar context-sensitivities:
``John smokes'' seems strange to say if he does so only once a year, but ``Mary climbs mountains'' seemingly is fine with the same frequency.
Both \emph{drinking moonshine}  and \emph{staring at the sun} will ``...make you go blind'', but the latter seems more dangerous than the former despite neither statement conveying precise risk information.

%Some have taken this evidence to suggest that generics cannot be understood in terms of how prevalent the property is, instead appealing to special mental relationships between certain kinds and certain properties.
%There have been a few formal models proposed for generic language, but none that have made precise, quantitative predictions about human endorsement patterns.

%The dominant view in psychology has been to focus on special conceptual relations that supposedly determine what makes the language the generalization true or false.
In this paper, we take seriously the idea that these statements convey generalization and look to \emph{probability}---the universal currency of belief---to formalize their meaning. 
Rather than have a fixed-meaning (e.g., how the quantifier ``most'' always means \emph{more than half}), we hypothesize that the generalizations have an uncertain meaning that must be resolved in context.
We formalize and embed this idea in a simple, rational model of communication and use it to address the fundamental outstanding question for the language of generalizations: What makes the sentences true or false?
Our answer is that true or false judgments are speaker decisions about whether or not to produce the context-sensitive generalization to a naive listener.
This is the first formal model to make precise quantitative predictions about human understanding of generalizations in language and we show how it is a sufficient general formulation to capture generalizations of diverse semantic types (categories, events, and causes).

We have previously received peer-review of a related manuscript submitted to top-ranking journals including \emph{Psychological Review}.
Those reviews raised a number of points regarding the presentation and generality of our work at the time, and highlighted the need for comparison to other models. 
The current manuscript differs from previous submissions in a number of substantial ways.
First, we have completely re-written the paper for clarity and focus, especially simplifying and streamlining the modeling sections.
Second, we have extended the work substantially from concerning only generalizations about categories (\emph{generic language}) to addressing two other semantic types: generalizations  about events and causes.
Across these three domains, we are able to explain endorsements with little-to-no domain-specific assumptions. 
Furthermore, each case study naturally lends methodological advantages to provide stricter tests of our hypothesis/model (e.g., with causal generalizations, we are able to manipulate---rather than merely measure---prior knowledge).
%The empirical and much of the philosophical literatures have focused on generalizations about categories, or generic language.
%Our previous manuscript also focused on categories; in this paper, however, we investigate three case studies: generalizations about categories, events, and causes. 
%Immediately we see the generality of our formal model: We are able to explain context-sensitive endorsements across these three domains without adopting domain-specific assumptions.
%Furthermore, each case study naturally lends methodological advantages to provide further tests of our model.
%With event generalizations, we are able to interrogate the underlying semantic scale, and find that speakers' subjective beliefs about what is likely to be the case in the future, and not the objective truth about what has been the case in the past, is what drives endorsements. 
%With causal generalizations, we are able to \emph{manipulate} (rather than merely measure) prior knowledge, and show it is in fact causally related to endorsement. 
Finally, we develop three alternative models: two linear models based on empirically-elicited statistical knowledge and one rich, cognitive model which lacks the uncertain meaning at the core of our hypothesis. 
These alternative models fail to account for key aspects of the context-sensitivity that is observed in the data and that our full model predicts.
%These alternative models and the empirical paradigms used provide strong tests of the theoretical contribution of our paper. 
%This novel approach for generalizations in language opens up a number of interesting questions about the nature of conceptual structure and its linguistic traces, which we discuss at the end of the paper.
Overall we feel that this paper represents a much improved, generalized and streamlined investigation of generalization in language, and a major breakthrough in computational modeling of language understanding.
(We would be happy to provide detailed responses to previous reviewers, if it would be useful; presently, we feel this would be of limited use given the major changes in content and presentation.)



%, which we use to make precise, quantitative predictions about sentence endorsement. 
% introduce the first formal model of generalizations in language and use it to make 
% this hypothesis in a probabilistic model of language understanding.  that the core 
%We adopt a formalism for the truth conditions of this language which has antecedents in formal semantics: a threshold function (e.g., the statement is true if the prevalence is greater than some criterion $\theta$).
%Attempts of this kind have been proposed before but none have been formalized into a model that makes predictions about human judgments. 
%The critical novelty of our approach is that the criterion $\theta$ is left underspecified in the language, and must be resolved in context.
%We instantiate this idea in a state-of-the-art probabilistic model of communication, and operationalize \emph{endorsement} (i.e., felicity or agreement with a statement) 



This work will be of interest to psychologists, as well as linguists, philosophers, and computer scientists interested in language and interaction.
The puzzles of generalizations in language are immediately compelling to a general audience, the more so because of their connections to cognitive development, propagation of stereotypes, and cultural transmission of knowledge.
Our solution represents the first formal model of generalization in language to be validated by quantitative agreement with human behavioral data.
It is not, however, uncontroversial: We champion a statistical view that has been often dismissed in the literature.
Indeed, in presentations around the world we have found this work to excite broad interest and engagement.

We recommend the following reviewers have expertise in the phenomenology of generic language, experimental studies of language use, and/or the Bayesian modeling tools we use:

\begin{itemize}
\item Susan A. Gelman, Department of Psychology, University of Michigan gelman@umich.edu
\item Greg N. Carlson, Department of Linguistics, University of Rochester carlson@ling.rochester.edu
\item Ted Gibson, Department of Brain and Cognitive Sciences, MIT egibson@mit.edu
\item Gerhard J{\"a}ger, Department of General Linguistics, University of T{\"u}bingen gerhard.jaeger@uni-tuebingen.de
%\item Mark Steedman, School of Informatics, University of Edinburgh  steedman@inf.ed.ac.uk
\item Naomi Feldman, Department of Linguistics, University of Maryland nhf@umd.edu
\item Amy Perfors, Department of Psychology, University of Adelaide amy.perfors@adelaide.edu.au
\item Klinton Bicknell, Department of Linguistics, Northwestern University kbicknell@northwestern.edu
\item Michael Franke, University of T{\"u}bingen, mchfranke@gmail.com
\item Tamar Kushnir, Department of Psychology, Cornell University, tk397@cornell.edu
%\item Michael Lee, Department of Cognitive Sciences, University of California, Irvine mdlee@uci.edu
\end{itemize}

We look forward to receiving your opinion.


\closing{Sincerely yours,}

\end{letter}

\end{document}

